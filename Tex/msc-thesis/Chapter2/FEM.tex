\chapter{A Finite Volume Method}
\label{ch2}

In order to compare the methods described later, which attempt to solve the problem described in \cref{ch1:sec2} using different PINN approaches, we need either an analytical solution of the drift-diffusion equation on each edge $e \in \mathcal{E}$ under the given initial and boundary conditions or a numerically very precisely generated solution. Since the former is out of the question, we use a finite volume method approach, which is considered to be very robust in general. In this section we derive this specific finite volume method and show what results can be achieved with it.  \\
Throughout this section we assume that the potential $V_e$ in \cref{Drift-Diffusion-equation} is affine linear and we write $d_e = \partial_x V_e$. Inside the set of edges incident to a vertex $\mathcal{E}_v$ we distinguish among $\mathcal{E}_v^{\text{in}} \ \coloneqq \{ e \in \mathcal{E}_v \colon e = ( \widetilde{v}, v ) \ \text{for some} \ \widetilde{v} \in \mathcal{V} \}$, the set of incoming edges in the vertex $v$, and $\mathcal{E}_v^{\text{out}} = \mathcal{E}_v \setminus \mathcal{E}_v^{\text{in}}$, the set of outgoing edges in the vertex $v$. \\
The control volumes of the method are defined as follows. To each edge $e \in \mathcal{E}$ we associate an equidistant grid of the parameter domain
\begin{equation*}
	0 = x^e_{1/2} < x^e_{3/2} <\ldots < x^e_{N_e+1/2} = \ell_e
\end{equation*}
with $h = x_{k+\frac{1}{2}} - x_{k-\frac{1}{2}}$, and introduce the intervals $I_k^e = (x_{k-1/2}, x_{k+1/2})$ for all $k=1, \ldots, N_e \in \mathbb{N}$. We introduce the following control volumes for our finite volume method,
\begin{itemize}
	\item the interior edge intervals $I_2^e, \ldots, I_{N_e-1}^e$ for
	each $e \in \mathcal{E}$,
	\item the vertex patch $I^v = \big(\cup_{e\in \mathcal{E}_v^{\text{in}}} I_{N_e}^n\big) \cup \big(\cup_{e\in \mathcal{E}_v^{\text{out}}} I_1^n\big)$ for
	each $v \in \mathcal{V}$,
\end{itemize}
and the semi-discrete approximation $\rho_h \colon [0,T] \times \Gamma \to \mathbb{R}$ of the problem \eqref{eq:Hamiltonian}--\eqref{eq:Dirichlet_conditions} can be expressed by the volume averages 
\begin{align*}	
	\rho^{v}= |I_v|^{-1}\int_{I^v}\rho_h\d x, \qquad
	\rho_k^{r,n} = |I_k^e|^{-1}\int_{I_k^e} \rho_h\d x,\ k=2,\ldots,N_e-1,
\end{align*}
for all $v \in \mathcal{V}$ and $e \in \mathcal{E}$. With the definition of the vertex patches $I^v$ we strongly enforce the continuity in each vertex $v$ of the graph. The flux function in a grid vertex $J(x^e_{k+1/2})$ is approximated by
\begin{equation*}
	J_{k+1/2}^e = -\varepsilon\,\frac{\rho_{k+1}^e-\rho_k^e}{h_e}
	+ F^e_{k+1/2},
\end{equation*}
where $F^e_{k+1/2}$ is a numerical flux for the transport term which must be chosen to guarantee consistency and stability of the numerical method. Here, we use the Lax-Friedrichs flux defined by
\begin{equation*}
	F^e_{k+1/2} = \frac{1}{2} (f(\rho_k^e) + f(\rho_{k+1}^e))\,d_e - \frac{\alpha_k^e}2
	(\rho_{k+1}^e - \rho_k^e) 
\end{equation*}
with some stabilization parameter $\alpha_k^e$. \\
Integration over all control volumes $I_k^e$, $k=2,\ldots,N_e-1$, and $I^v$, $v\in \mathcal{V}$, yields the equation
\begin{align*}
	h_e\,\partial_t \rho_k^e - J^e_{k-1/2} + J^e_{k+1/2} &= 0 \\
	\sum_{e\in \mathcal{E}_v} h_e\,\partial_t \rho^v
	- \sum_{e\in \mathcal{E}_v^{\text{in}}} J^e_{N_e-1/2} 
	+ \sum_{e\in \mathcal{E}_v^{\text{out}}} J^e_{3/2} &= 0
\end{align*}
and insertion of the approximate flux function yields
\begin{align*}
	h_e\,\partial_t \rho_k^e + \varepsilon\,\frac{-\rho_{k-1}^e +
	2\rho_k^e - \rho_{k+1}^e}{h_e} - F_{k-1/2}^e + F_{k+1/2}^e &= 0
	  \\
	\sum_{e\in \mathcal{E}_v} h_e\partial_t\rho^v
	+ \sum_{e\in \mathcal{E}_v^{\text{in}}}
	\left(\varepsilon\,\frac{\rho^v-\rho_{N_e-1}^e}{h_e} -
	F^e_{N_e-1/2}\right)
	+ \sum_{e\in \mathcal{E}_v^{\text{out}}}
	\left(\varepsilon\,\frac{\rho^v-\rho_2^e}{h_e} + F^e_{3/2}\right)
	&= 0
\end{align*}
for all $k = 2,\ldots,N_e-1$, $e\in \mathcal{E}$ and $v \in \mathcal{V}$. We note that the integral over the cell patch is realized edge-wise. The accumulated contributions from the integration-by-parts formula at the vertex $v$ vanish due to the Kirchhoff-Neumann vertex conditions \eqref{eq:Kirchhoff_Neumann_condition}. Incorporate inflow- and outflow boundary conditions. \\
To derive a fully-discrete scheme, we integrate the above equations over the intervals $[t_n, t_{n+1})$, $n=1, \ldots, N_t-1$ of an equidistant time grid $\{ t_n \coloneqq \tau\,n \colon n = 0, \ldots, N_t \}$ with the temporal discretization parameter $\tau = T / N_t$. We use appropriate quadrature formulas, more precisely, the rectangle rule using the evaluation point $t_{n+1}$ for the diffusion terms and the point $t_n$ for the transport terms. In other words, we evaluate the diffusion terms implicitly and the (non-linear) transport term explicitly. Introducing the notation $\rho^{v,n} = \rho^v|_{[t_n,t_{n+1})}$ and $\rho_k^{e,n} = \rho_k^e|_{[t_n,t_{n+1})}$ yields the following set of equations: \textcolor{red}{Set $\rho^{e,n+1}_{N_e} = \rho^{v,n+1}$ and the same for $\rho^{e,n+1}_1$.}
\begin{subequations}
    \label{eq:fully_discrete_fvm}
    \begin{align}
        h_e\,\partial_t \rho_k^{e,n+1} + \varepsilon\,\frac{-\rho_{k-1}^{e,n+1} +
        2\rho_k^{e,n+1} - \rho_{k+1}^{e,n+1}}{h_e} - F_{k-1/2}^{e,n} +
        F_{k+1/2}^{e,n} &= 0 \\	
        \sum_{e\in \mathcal{E}_v} h_e\partial_t\rho^{v,n+1}
        + \sum_{e\in \mathcal{E}_v^{\text{in}}}
        \left(\varepsilon\,\frac{\rho^{v,n+1}-\rho_{N_e-1}^{e,n+1}}{h_e} -
        F^{e,n}_{N_e-1/2}\right)
        + \sum_{e\in \mathcal{E}_v^{\text{out}}}
        \left(\varepsilon\,\frac{\rho^{v,n+1}-\rho_2^{e,n+1}}{h_e} + F^{e,n}_{3/2}\right)
        &= 0
    \end{align}
\end{subequations}
for all $k = 2, \ldots, N_e-1$, $e \in \mathcal{E}$, $v \in \mathcal{V}$ and $n = 0, \ldots, N_t - 1$. The initial data are established by
\begin{equation*}
	\rho_k^{e,0}=\pi_{I_k^e}(\rho_0),\qquad \rho^{v,0} = \pi_{I^v}(\rho_0),
\end{equation*}
where $\pi_M$ denotes the $L^2$-projection onto the constant functions on a subset $M \subset \Gamma$. Note that this set of equations is linear in the unknowns in the new time point $\rho_k^{e,n+1}$, $k = 2, \ldots, N_e-1$, $ e \in \mathcal{E}$ and $\rho^{v,n+1}$, $v \in \mathcal{V}$. The fully-discrete approximation $\rho_{\tau\,h}\colon [0,T]\times \Gamma\to \mathbb{R}$ then reads
\begin{align*}
	\rho_{\tau\,h}(t,x) &= \rho_h^n(x)\ \text{if}\ t\in [t_n,t_{n+1}),\\
	\text{with}\qquad \rho_h^n(x) &= \rho^{v,n},\ x\in I^v,\quad \rho_h^n(x)
	= \rho_k^{e,n}, x\in I_k^e.
\end{align*}
This numerical scheme preserves the following important properties known from the continuous setting.
\begin{theorem}
    The solution of \eqref{eq:fully_discrete_fvm}, $\rho_{\tau\,h}$ satisfies the following properties:
    \begin{enumerate}[label=\roman*)]
        \item The scheme is mass conserving, i.e., if $\alpha_v\equiv \beta_v\equiv 0$ for all $v\in \mathcal{V}_{\mathcal{D}}$, then there holds
        \begin{equation*}
            \int_\Gamma\rho_h^n\d x = \int_\Gamma\rho_h^0\d x\qquad\forall n=1,\ldots,N_t.
        \end{equation*}
        \item The scheme is bound-preserving, i.e., there holds
        \begin{equation*}
            \rho_{\tau\,h}(t,x)\in [0,1]\qquad \forall t\in [0,T],
            x\in \Gamma.
        \end{equation*}
    \end{enumerate}
\end{theorem}

