\chapter{Conclusion}
\label{ch5}

This work has set itself the objective of adapting the $PINN$ setup introduced in \cite{RaissiPerdikarisKarniadakisPart1:2017} for the case where the partial differential equation is a drift-diffusion equation posed on a metric graph. First, the basics for metric graphs were defined, the approximation problem defined by considering the drift-diffusion equation on each edge of a metric graph was explained, neural networks were introduced in general and their use in the $PINN$ setup was discussed as well as the idea behind using a $PINN$ approach for approximating the solution of a differential equation. In this thesis we constructed a cost function that incorporates the deviation of the corresponding neural network from the solution of the differential equation on each edge via a residual network as well as the deviation of the corresponding neural network of the initial condition on each edge as well as the deviation of the corresponding neural network from the vertex conditions on each vertex via several misfit terms. We have demonstrated in numerical experiments that this resulting $PINN$ approach offers an alternative to the commonly used $FVM$. Furthermore, by using different neural networks for this approach, the numerical experiments showed that the use of one $FNN$ for each edge of the metric graph resulted in the smallest error for our approximation problem with respect to the values generated by the $FVM$. We have presented explicit methods for computing the first and second order derivatives for the in the numerical experiments used neural networks and we compared the required time by these in \lstinline!Python 3.8.8! implemented methods with the required time by using $AD$ in \lstinline!Python 3.8.8!, which is provided by \lstinline!tf.GradientTape! from \lstinline!Tensorflow!. It turned out that the explicit methods presented in this thesis are not an alternative to $AD$ for our $PINN$ approach. \\
During originating this thesis, new research questions arose. We could not include all of them in this thesis \cite[p.~65]{Wagner20}. For instance, the question arises whether a different $PINN$ approach with a different cost function achieves a better approximation quality or at least achieves comparable results in a shorter required time. This cost function could, for example, be defined individually for each edge or for each vertex. Also, the question of the most suitable neural network for this $PINN$ approach has not yet been clarified definitively, for which the use of hyperparameter optimisation could provide useful results. Nevertheless, these and other aspects remain to be addressed for the future.  \\


