\section{Metric Graphs}

We define a graph $\Gamma$ as an ordered pair $(\mathcal{V}, \mathcal{E})$, where $\mathcal{V} = \{v_i\}$ is a finite set of points, which we call vertices, and $\mathcal{E} = \{e_j\}$ is a set of segments connecting some of the vertices, which we call edges. In the following we will use the notation $E \coloneqq \left\lvert \mathcal{E} \right\rvert$ for the number of edges and $V \coloneqq \left\lvert \mathcal{V} \right\rvert$ for the number of vertices. \\
Each edge $e \in \mathcal{E}$ can be identified with a pair $(v, w)$ of vertices $v, w \in \mathcal{V}$, which it connects. A graph $\Gamma = (\mathcal{V}, \mathcal{E})$ is called directed graph, if each of its edges $e \in \mathcal{E}$ is assigned a direction, which means that each edge can only be followed in one direction. In this case the order of the pair of vertices $(v, w)$ describing an edge $e$ is important, so that the set of edges $\mathcal{E}$ can be uniquely described as a set of ordered pairs $\mathcal{E} = \{e_i\}_{i = 1, \ldots, E} = \{(v^{o}_{i}, v^{t}_{i})\}_{i = 1, \ldots, E}$, where the first vertex $v^{o}_{i}$ is called the origin vertex and the second vertex $v^{t}_{i}$ is called the terminal vertex of the corresponding edge $e_i$. A graph $\Gamma = (\mathcal{V}, \mathcal{E})$ is non-directed if each of its edges $e \in \mathcal{E}$ can be followed in both directions. In this case, the order of the two connected vertices describing an edge is unimportant, i.e. $e = (v, w) = (w, v)$, $v, w \in \mathcal{V}$. \\
A vertex $w \in \mathcal{V}$ is adjacent to a vertex $v \in \mathcal{V}$, denoted by $v \sim w$, if a suitable edge $e \in \mathcal{E}$ exists, so that $w$ can be reached from $v$ via this edge $e$. In the following we assume that a graph has no loops, which are edges that connect a vertex $v \in \mathcal{V}$ to itself, i.e. $v \sim v$, and no multi-edges, which are several equal edges between two vertices, i.e. $e_1, e_2 \in \mathcal{E}$ and $e_1= e_2$. A graph $\Gamma = (\mathcal{V}, \mathcal{E})$ with $\mathcal{V} = \{v_i\}_{i = 1, \ldots, V}$ is fully specified by its $V \times V$ adjacency matrix $A^{\Gamma}$. The elements of the adjacency matrix are given by
\begin{equation}
    \label{adjacency matrix}
    A^{\Gamma}_{i, j}= \begin{cases} 1 & \text { if } v_i \sim v_j \\ 0 & \text { otherwise. } \end{cases}
\end{equation}
One sees immediately that the adjacency matrix of an undirected graph is symmetric, since a vertex $v \in \mathcal{V}$ is adjacent to another vertex $w \in \mathcal{V}$ exactly when the other way round is also true, i.e. $v \sim w \Leftrightarrow w \sim v$. \\
The degree $d_{v_i}$ of a vertex $v_i \in \mathcal{V}$ is the number of edges emanating from it, i.e. $d_{v_i} = \sum_{v_j \in \mathcal{V}} A_{i, j}$. All degrees are assumed to be finite. We will denote by $D^{\Gamma}$ the degree matrix of the graph, which is a diagonal $V \times V$ matrix with the entries
\begin{equation}
    \label{degree matrix}
    D^{\Gamma}_{i, j} = d_{v_i} \, \delta_{v_j, v_i}
\end{equation}
where $\delta_{v_j, v_i}$ is the Kronecker delta. \\
A vertex is incident to an edge, denoted by $v \in e$, if the vertex $v \in \mathcal{V}$ is one of the two vertices the edge $e \in \mathcal{E}$ connects. We define for an undirected graph $\Gamma = (\mathcal{V}, \mathcal{E})$ with $\mathcal{V} = \{v_i\}_{i = 1, \ldots, V}$ and $\mathcal{E} = \{e_i\}_{i = 1, \ldots, E}$ the $V \times E$ incidence matrix $B^{\Gamma}$ by  
\begin{equation}
    \label{incidence matrix undirected}
    B^{\Gamma}_{i, j}= \begin{cases} 1 & \text { if } v_i \in e_j \\ 0 & \text { otherwise } \end{cases}
\end{equation}
and for a directed graph $\Gamma = (\mathcal{V}, \mathcal{E})$ with $\mathcal{V} = \{v_i\}_{i = 1, \ldots, V}$ and $\mathcal{E} = \{e_i\}_{i = 1, \ldots, E}$ we define the $V \times E$ incidence matrix $B^{\Gamma}$ by 
\begin{equation}
    \label{incidence matrix directed}
    B^{\Gamma}_{i, j}= \begin{cases} 1 & \text { if } e_j = (v_i, \cdot) \\ -1 & \text { if } e_j = (\cdot, v_i) \\ 0 & \text { if } v_i \notin e_j. \end{cases}
\end{equation}

% We also denote by $\mathcal{E}_v$ the set of all edges incident to the vertex $v$ (i.e., containing $v$). It is assumed that the degree (valence) dv = |Ev| of any vertex v is finite and positive. We hence exclude vertices with no edges coming in or going out.

So far we have considered graphs as discrete combinatorial objects. From now on we consider graphs as 1-dimensional complexes and the edges will be treated as 1-dimensional segments. Roughly speaking, we now imagine the edges $\mathcal{E}$ of a graph $\Gamma = (\mathcal{V}, \mathcal{E})$ not as abstract relations between the vertices $\mathcal{V}$, but rather as physical “wires” connecting them. For that we add a structure that equips $\Gamma$ with a topology and metric. 

\begin{definition}\label{metric graph}
    A graph $\Gamma = (\mathcal{V}, \mathcal{E})$ is said to be a metric graph, if each of its edges $e \in \mathcal{E}$ is assigned a positive length $l_e > 0$.
\end{definition}

%Having the length assigned, an edge e will be identified with a finite or infinite segment [0, le] of the real line with the natural coordinate xe along it. In most cases we will drop the subscript in the coordinate and call it x, which should not lead to any confusion. This enables one to interpret the graph Γ as a topological space (simplicial complex) that is the union of all edges where the ends corresponding to the same vertex are identified

