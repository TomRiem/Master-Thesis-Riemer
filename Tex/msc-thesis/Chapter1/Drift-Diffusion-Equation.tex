\section{Drift-Diffusion Equations on Metric Graphs}
\label{ch1:sec2}

In this section, we introduce a set of differential equations on a metric graph by which the density of cars on a compact road network are modelled. This provides us with an approximation problem, which we attempt to solve later in this thesis using various methods. \\
We imagine a compact road network, where we mean a finite number of finitely long roads connected by a finite number of junctions. We note that in a realistic road network there also exist roads which simply end, i.e. roads which end in an junction to which no other roads are connected, and also one-way roads exist between two junctions. We thus identify a compact road network with a compact directed metric graph, where obviously the roads are identified by edges and the junctions by vertices. An example of such a compact road network is illustrated in Figure 1 and the corresponding graph is illustrated in Figure 2. \\



It links the temporal change in the spatial density at which this conservation variable is present at a point to the spatial change in its flux density.

traffic flow on networks The main inspiration is that of understanding traffic behavior in urban context in order to answer

In the case of complex dynamics observed within the network, metric graphs provide a natural representation as both the edge length is incorporated into the modeling as well as a partial differential equation defined on the graph

This corresponds to the ideas originally introduced in the context of fluid models. A very prominent fluidlike traffic flow model is the Lighthill-Whitham-Richards model (LWR model) by Lighthill and Whitham (1955) and Richards (1956), w

Classically, the network models of transportation systems are assumed to be static, but these models do not allow a correct simulation of heavily congested urban road networks. For this reason, traffic engineers have been studying dynamic traffic assignment or within-day models, thus rendering necessary the use of time advancing mathematical models (traffic simulation models). These models, principally created from static network traffic assignments, can be roughly classified in microscopic, mesoscopic and macroscopic (see [10] and the references therein). The main problems of this approach consist in the fact that it does not properly reproduce the backward propagation of shocks and in the difficulty of collecting experimental data to test the models. Various other ideas have been developed by researchers studying traffic from other perspectives, see for instance [9, 38, 63, 68, 89, 91, 106]. In many cases, the attention was focused on a single road or on small portions of an urban network

In the 1950s James Lighthill and Gerald Whitham, two experts in fluiddynamics, (and independently P. Richards) thought that the equations describing the flow of water could also describe the flow of car traffic. These equations in fluid dynamics are a set of partial differential equations known as the Euler or Navier-Stokes equations, expressing the conservation of mass, momentum and energy. The basic idea is to look at large scales so to consider cars as small particles and their density as the main quantity to be considered.



In any case, it is reasonable to assume the conservation of the number of cars, thus leading again to a conservation law. As traffic jams display sharp discontinuities, there is a correspondence between traffic jams and shock waves. Therefore, fluid-dynamic models for traffic flow seem the most appropriate to detect some phenomena as shocks formation and propagation on roads, since solutions can develop discontinuities in a finite time even starting from smooth initial data (


%erster teil diffussion
%zweiter Teil transport
%$\rho$ ist dichte. Nie über eins
%Kirchhoff: alles was in den Knoten reinfließt muss auch rausfließen
% Kirchhoff: We note that the derivative is taken into the outgoing direction.
%operator nicht selbstadjungiert
% graph muss gerichtet sein, ansosnten erste ableitung nicht möglich
%initial conditions übernehmen
% Analytische Lösung nicht möglich
% 3.1.5 Lighthill-Whitham-Richards Model with Viscosity


\begin{equation} 
    \label{Drift-Diffusion-equation}
    \partial_t \rho_b (t,x) = \partial_x (\varepsilon \partial_x \rho_b (t,x) - f(\rho_b (t,x) ) \partial_x V_b (t,x)).
\end{equation}


The model is posed on a directed metric graph $\Gamma = (\mathcal{V}, \mathcal{E})$, where each edge $b \in \mathcal{E} = \mathcal{B}$ is equipped with a length $\ell_b > 0$ and a differential operator
\begin{equation} 
    \label{eq:Hamiltonian}
    \mathcal{H} [\rho_b] (t,x) \coloneqq \partial_t \rho_b (t,x)  - \partial_x (\varepsilon \partial_x \rho_b (t,x) + f(\rho_b (t,x) ) \partial_x V_b (t,x)).
\end{equation}
From now on we denote the ordered pair of vertices connected by a bond $b$ by $(v^{o}_b, v^{t}_b)$ and introduce a normal vector $n_b(v)$ to each bond defined via $n_b(v^{o}_b) = -1$ and $n_b(v^{t}_b) = 1$. Further we denote the edge set incident to a vertex $v \in \mathcal{V}$ by $\mathcal{E}_v$. In \cref{eq:Hamiltonian}, $\rho_e \colon (0,T) \times e \to \mathbb{R}_{+}$ describes, on each edge, the concentration of some quantity while $V_e \colon (0,T) \times e \to \mathbb{R}_{+}$ is a given potential, that may vary from edge to edge, and $\varepsilon > 0$ a given constant, typically small. The function $f \colon \mathbb{R}_{+} \to \mathbb{R}_{+}$ is called mobility, satisfies $f(0) = f(1) = 0$ and its simplest choice is $f(\rho_e) = \rho_e$, i.e. linear transport. However, in many applications, the density is now allowed to exceed a maximal value (e.g. due to finite size effects). If this value is scaled to one, a choice of $f$ that obeys $f(1) = 0$ will ensure that solutions to \eqref{eq:Hamiltonian} will satisfy this bound for all times. A typical choice for $f$ will be $f(\rho_e) = \rho_e(1-\rho_e)$ in the following. \\
To make \cref{eq:Hamiltonian} a well-posed problem, we need to add initial-conditions as well as coupling conditions on the vertices. On the set of interior vertices $v \in \mathcal{V}_\mathcal{K} \subset \mathcal{V}$, which are vertices that are incident to at least one incoming edge and at least one outgoing edge (i.e.$\forall v \in \mathcal{V}_\mathcal{K} \; \exists \ e_1, e_2 \in \mathcal{E}$ such that $v^{t}_{e_1} = v$ and $v^{o}_{e_2} = v$), we apply homogeneous Kirchhoff-Neumann coupling conditions, i.e., there holds
\begin{equation}
    \label{eq:Kirchhoff_Neumann_condition}
    \sum_{e\in \mathcal{E}_v} J_e(t,v) n_e (v)=0,
\end{equation}
where $J_e \colon (0,T) \times e \to \mathbb{R}$ is defined as the flux in a spatial point $x \in e$ via 
\begin{equation} 
    \label{eq:flux} 
    J_e(t,x) \coloneqq - \varepsilon \partial_x \rho_e (t, x) + f(\rho_e(t, x)) \partial_x V_e(t, x).
\end{equation}
Additionally, we ask the solution to be continuous on the set of interior vertices, i.e. 
\begin{equation}
    \label{continuous on vertices}
    \rho_e(v) = \rho_{e'}(v) \quad \text{ for all }v \in \mathcal{V}_\mathcal{K},\; e,\,e' \in \mathcal{E}_v
\end{equation}
with $\mathcal{E}_v$ the edge set incident to the vertex $v$. \\
On the set of exterior vertices $v \in \mathcal{V}_\mathcal{D} \coloneqq \mathcal{V} \setminus \mathcal{V}_\mathcal{K}$, which are vertices that are incident to either only one incoming edge or only one outgoing edge (i.e. $\forall v \in \mathcal{V}_\mathcal{D} \; \exists! \ e \in \mathcal{E}$ such that either $v^{t}_{e} = v$ or $v^{o}_{e} = v$), the solution $\rho$ should fulfills the flux boundary conditions
\begin{equation}
    \label{eq:Dirichlet_conditions}
    \sum_{e\in \mathcal{E}_v}J_e(v) n_e (v)=-\alpha_v(t) (1-\rho_v) + \beta_v(t) \rho_v,\ \text{for all}\ v \in \mathcal{V}_\mathcal{D}, e \in \mathcal{E}_v,
\end{equation}
where $\alpha_v \colon (0,T) \to \mathbb{R}_{+}, \, \beta_v \colon (0,T) \to \mathbb{R}_{+}$, ${v \in \mathcal{V}_\mathcal{D}}$ are functions prescribing the rate of influx of mass into the network ($\alpha_v$) as well as the velocity of mass leaving the network ($\beta_v$) at the boundary vertices. Note that this choice ensures that the bounds $0 \le \rho_e \le 1$ are preserved. In typical situations, boundary vertices are either of influx- or of outflux type, i.e. $\alpha_v \beta_v = 0$ for all $v \in \mathcal{V}_\mathcal{D}$. \\
The Kirchhoff-Neumann conditions are the natural boundary conditions for the differential operator \eqref{eq:Hamiltonian}, as they ensure that mass enters or leaves the system only via the boundary vertices $\mathcal{V}_\mathcal{D}$ for which either $\alpha_v$ or $\beta_v$ is positive. \\
Having introduced the complete continuous model, we state the following existence and uniqueness result. Details on the exact definition of the spaces and a sketch of the proof can be found in Section 

\begin{equation}
    \label{eq:initial_conditions}
    \rho_e(0,x) = \rho_{e, 0}(x) 
\end{equation}


\begin{theorem} 
    Given initial data $\rho_0 \in L^2(\Gamma)$ s.t. $0 \le \rho_0 \le 1$ a.e. on $\mathcal{E}$, there exists a unique weak solution $\rho \in L^2(0,T; H^1(\Gamma)) \cap H^1(0,T; (H^1)^*(\Gamma))$ s.t.
	\begin{align*}
		\sum_{e \in \mathcal{E}} \left(\int_e  \partial_t \rho_e \mathcal{V}arphi_e \;dx + \int_e \partial_x \rho_e\partial_x \mathcal{V}arphi_e \;dx\right) + \sum_{v \in \mathcal{V}_D} (-\alpha_v(t) (1-\rho_v) + \beta_v(t) \rho_v)\mathcal{V}arphi(v) = 0,
	\end{align*}
	for all testfunction $\mathcal{V}arphi \in H^1(\Gamma)$.
\end{theorem}

