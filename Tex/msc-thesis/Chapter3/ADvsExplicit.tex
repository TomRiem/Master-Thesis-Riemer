\section{Automatic differentiation vs explicit derivatives}
\label{ch3:sec3}

In this section we show how the neural networks used in the previous numerical tests can be explicitly differentiated so that the use of automatic differentiation in a PINN can be omitted. We compare the time to convergence and the computational resources required for a PINN programmed in Python with explicit derivatives with the results previously obtained for a PINN that of course uses automatic differentiation. This aims to show whether an explicit calculation of the derivatives for our problem brings computational advantages with it and whether the performance of a PINN can be improved as a result. \\
Let us first consider the case of the feed-forward neural network. In this work we used two FNNs with different topologies. First, an FNN was used in section 1 for all edges of the graph and then an FNN was used for each individual edge of the graph. We are interested in finding an explicit formula for the gradient of the output used for an individual edge of one of these two neural networks. For this purpose, we first calculate the first-order derivative with respect to a vector of a general FNN $f_{\theta}$ with $L$ layers. In the following, $f_{\theta}(x) \in \mathbb{R}^{n_L}$ denotes the model prediction of this neural network for an input value $x = x^0 \in \mathbb{R}^{n_0}$, which is defined by 
\begin{equation}
    \label{model prediction}
    f_{\theta}(x) = x^{L} = f^{L}_{\theta_L}(x^{L-1}) = \sigma_{L} (W^{L} x^{L-1} + b^{L}) \in \mathbb{R}^{n_L} 
\end{equation}
and 
\begin{equation*}
    x^{l} = \sigma_{l} (W^{l} x^{l-1} + b^{l}) \in \mathbb{R}^{n_l}, 
\end{equation*}
for $0 \leq l \leq L-1$. We are interested in the first order derivative of $f_{\theta} \colon \mathbb{R}^{n_0} \to \mathbb{R}^{n_L}$ with respect to the input value $x = x^0 \in \mathcal{R}^{n_0}$, which can be represented by the Jacobian matrix of $f_{\theta}$
\begin{equation*}
    \frac{\mathrm{d}}{\mathrm{d} \ x} f_{\theta}(x) = \mathrm{}J[f_{\theta}](x)= \begin{pmatrix} \dfrac{\partial f_{\theta}(x)}{\partial x_{1}} & \cdots & \dfrac{\partial f_{\theta}(x)}{\partial x_{n_0}} \end{pmatrix} = \begin{pmatrix} \nabla f_{\theta, 1}(x)^{\mathrm{T}} \\ \vdots \\  \nabla f_{\theta, n_L}(x)^{\mathrm{T}} \end{pmatrix} = \begin{pmatrix} \dfrac{\partial f_{\theta, 1}(x)}{\partial x_{1}} & \cdots & \dfrac{\partial f_{\theta, 1}(x)}{\partial x_{n_0}} \\ \vdots & \ddots & \vdots \\ \dfrac{\partial f_{\theta, n_L}(x)}{\partial x_{1}} & \cdots & \dfrac{\partial f_{\theta, n_L}(x)}{\partial x_{n_0}} \end{pmatrix} 
\end{equation*}
where $f_{\theta, i}(x)$ is the $i$-th component of the output $f_{\theta}(x) \in \mathbb{R}^{n_L}$ and $\nabla f_{\theta, n_L}(x)^{\mathrm{T}}$ is the transpose of the gradient of $f_{\theta, i}(x)$. \\
To calculate the Jacobi matrix of $f_{\theta}$ we use the chain rule several times in order to differentiate the network layer by layer. For that we denote the Jacobian matrix of the activation function $\sigma_{l} \colon \mathbb{R}^{n_l} \to \mathbb{R}^{n_l}$ of the $l$-th layer with respect to the activation $a^l(x^{l-1}) = W^{l} x^{l-1} + b^{l}$ of the $l$-th layer, with
\begin{equation*}
    \frac{\mathrm{d}}{\mathrm{d} \ a^{l}} \ \sigma_{l} (a^l) = D^{l} \in \mathbb{R}^{n_l \times n_l}.
\end{equation*}
We remind that the activation function of a layer is defined component-wise, thus the Jacobian matrix is a diagonal matrix and its diagonal entries are given by
\begin{equation*}
    D_{i, i}^{l} = {\sigma_{l}}^{\prime} (a^{l}(x^{l-1}))_i = {\sigma_{l}}^{\prime} (a^{l}(x^{l-1})_i).
\end{equation*} 
We can now differentiate the network using the chain rule
\begin{align*}
    \frac{\mathrm{d}}{\mathrm{d} \ x} f_{\theta}(x) & = \frac{\mathrm{d}}{\mathrm{d} \ a^{L}} \ \sigma_{L} (a^{L}(x^{L-1})) = \\
    & = D^L \cdot \frac{\mathrm{d}}{\mathrm{d} \ x^{L-1}} \ a^{L}(x^{L-1}) = D^L \cdot \frac{\mathrm{d}}{\mathrm{d} \ x^{L-1}} \ (W^{L} x^{L-1} + b^{L}) = \\
    & = D^L \cdot W^L \cdot \frac{\mathrm{d}}{\mathrm{d} \ a^{L_1}} \ \sigma_{L-1} (a^{L-1}(x^{L-2})) = \\
    & = \ldots = \\
    & = D^L \cdot W^L \cdot D^{L-1} \cdot \ldots \cdot W^2 \cdot D^1 \cdot \frac{\mathrm{d}}{\mathrm{d} \ x^{0}} \ (W^{1} x^{0} + b^{1}) = \\
    & = D^L \cdot W^L \cdot D^{L-1} \cdot \ldots \cdot W^2 \cdot D^1 \cdot W^{1} \in \mathbb{R}^{n_L \times n_0}.
\end{align*}

Let us return to our case where an FNN rho approximates the solution of the drift-diffusion equation on an edge. In the following we use the vector z, where its first component is the time and the second component is the position on an edge. The gradient is due to equation 3 and equation 4 given by 
\begin{equation}
    \begin{split}
        \nabla_{z}  \ \rho_{\theta_e}(z) & = \left(\nabla_{z} \rho_{\theta_e}(z)^{\mathrm{T}} \right)^{\mathrm{T}} = \left(D^L \cdot W^L \cdot D^{L-1} \cdot \ldots \cdot W^2 \cdot D^1 \cdot W^{1} \right)^{\mathrm{T}} = \\
        & = {W^{1}}^{\mathrm{T}} \cdot D^{1} \cdot {W^{2}}^{\mathrm{T}} \cdot \ldots \cdot {W^{L}}^{\mathrm{T}}  \cdot  D^{L}. 
    \end{split}
\end{equation}
The partial derivatives can be obtained by 
\begin{equation*}
    \begin{split}
        \partial_t \rho_{\theta_e}(t,x) = \nabla_{z} \ \rho_{\theta_e}(z)_1 \quad & \text{and} \quad \partial_x \rho_{\theta_e}(t,x) = \nabla_{z} \ \rho_{\theta_e}(z)_2 \\
        \partial_t \rho_{\theta_e}(t,x) = {\nabla_{z} \ \rho_{\theta_e}(z)}^{\mathrm{T}} \begin{pmatrix} 1 \\ 0 \end{pmatrix} \quad & \text{and} \quad \partial_x \rho_{\theta_e}(t,x) = {\nabla_{z} \ \rho_{\theta_e}(z)}^{\mathrm{T}} \begin{pmatrix} 0 \\ 1 \end{pmatrix}.
    \end{split}
\end{equation*}


To compute the gradient in a computer program, one proceeds similarly to backpropagation by first evaluating the network at a point, i.e. propagating information forwards through the network, storing relevant values, and then propagating the so-called error backwards through the network by applying the chain rule from the last layer to the first layer in reverse. The following iterative computation can be used for that purpose
\begin{align*}
    \delta^{L} & = {W^{L}}^{\mathrm{T}} \mathrm{diag}({\sigma_{L}}^{\prime}(W^{L} a^{L-1}(s) + b^{L})) = {W^{L}}^{\mathrm{T}} D^{L} \in \mathbb{R}^{n_{L-1}} \\
    \delta^{l} & = {W^{l}}^{\mathrm{T}} D^{l} \cdot \, \delta^{[l+1]} \, \cdot \, \ldots \,  \cdot \, \delta^{L} \in \mathbb{R}^{n_{l-1}} \quad \text{for} \quad l = 2, \ldots, L-1
\end{align*}


We set 

\begin{equation*}
    H^{l} = \mathrm{diag}({\sigma_{l}}^{\prime \prime}(W^{l} a^{l-1}(s) + b^{l})) \in \mathbb{R}^{n_l \times n_l}, \quad (H^{l})_{i, i} = {\sigma_{l}}^{\prime \prime} (z_{i}^{l})
\end{equation*}

\begin{align*}
    \nabla^{2} f_{\theta}(s) & = \eta^{[1]} \in \mathbb{R}^{2 \times 2} \\
    \eta^{[1]} & = {W^{[1]}}^{\mathrm{T}} \left( \mathrm{diag}(H^{[1]} \delta^{[2]}) + D^{[1]} \eta^{[2]} D^{[1]} \right) W^{[1]} \\
    & \vdots \\
    \eta^{l} & = {W^{l}}^{\mathrm{T}} \left( \mathrm{diag}(H^{l} \delta^{[l+1]}) + D^{l} \eta^{[l+1]} D^{l} \right) W^{l} \in \mathbb{R}^{n_{l-1} \times n_{l-1}} \\
    & \vdots \\
    \eta^{L} & = {W^{L}}^{\mathrm{T}} H^{L} W^{L} \in \mathbb{R}^{n_{L-1} \times n_{L-1}}
\end{align*} 


\begin{algorithm}[H]
    \caption{Computation of the gradient and a Hessian of an L-layer feed-forward neural network.}
    \begin{algorithmic}[1]
        \State \textbf{Input:} vector $z^0 \in \mathbb{R}^{n_0}$; learnable parameters $\theta = (\{ W^l \}_{l = 1, \ldots, L}, \{ b^l \}_{l = 1, \ldots, L})$ with $W^l \in \mathbb{R}^{n_l \times n_{l-1}}$ and $b^l \in \mathbb{R}^{n_l}$ for $l = 1, \ldots, L$, where $n_L = 1$; activation functions $\{ b^l \}_{l = 1, \ldots, L}$ and their first order derivative up to order 2 of a FNN $f$
        \For{$ l = 1, \ldots, L$}
            \State Set $z^l = \sigma_l(W^l z^{l-1} + b^l)$
        \EndFor
        \State Set $\delta^{L} = {W^{L}}^{\mathrm{T}} \mathrm{diag}({\sigma_{L}}^{\prime}(W^{L} z^{L-1} + b^{L}))$.
        \State Set $\eta^{L}$.
        \For{$ l = 1, \ldots, L-1$}
            \State Set $\delta^{l} = {W^{l}}^{\mathrm{T}} \mathrm{diag}({\sigma_{l}}^{\prime}(W^{l} z^{l-1} + b^{l})) \cdot \delta^{l+1}$.
            \State Set $\eta^{l}$.
        \EndFor
        \State \textbf{Input:} $z^L$, $\delta^1$, $\eta^1$.
    \end{algorithmic}
\end{algorithm}




  






We note that for $A \in \mathbb{R}^{m \times n}$, $b \in \mathbb{R}^{m}$, $c, d \in \mathbb{R}^{n}$

\begin{align*}
    d \, \odot \, c &= c \, \odot \, d \in \mathbb{R}^{n} \\
    \mathrm{diag}(b) \, \cdot \, A &= b \, \odot \, A \in \mathbb{R}^{m \times n} \\
    A \, \cdot \, \mathrm{diag}(c) &= A \, \odot \, c^{\mathrm{T}} \in \mathbb{R}^{m \times n} \\
    \mathrm{diag}(b) \, \cdot \, A \, \cdot \, \mathrm{diag}(c) &= b \, \odot \, A \, \odot \, c^{\mathrm{T}} \in \mathbb{R}^{m \times n} \\
    \mathrm{diag}(c) \, \cdot \, d &= c \, \odot \, d \in \mathbb{R}^{n} \\
    c^{\mathrm{T}} \, \cdot \, \mathrm{diag}(d) &= c^{\mathrm{T}} \, \odot \, d^{\mathrm{T}} \in \mathbb{R}^{1 \times n} \\
\end{align*}







\subsection{Backpropagation in ResNet}



where $W^{[1]} \in \mathbb{R}^{m \times 2}$ and $b^{[1]} \in \mathbb{R}^{m}$.

\begin{equation*}
    a^{l}(x) = a^{l-1}(x) + h \, \sigma_{l} (z^{l}(x)) = a^{l-1}(x) + h \, \sigma_{l} (W^{l} a^{l-1}(x) + b^{l}) \in \mathbb{R}^{m}, 
\end{equation*}

for $l = 2, \ldots, L$, where $W^{[2]}, \ldots, W^{L} \in \mathbb{R}^{m \times m}$ and $b^{[2]}, \ldots, b^{L} \in \mathbb{R}^{m}$.

\begin{equation*}
    f_{\theta}(x) = f_{\theta}(s) = w^{\mathrm{T}} a^{L}(s) + \frac{1}{2} s^{\mathrm{T}} A s + c^{\mathrm{T}} s 
\end{equation*}

where $s = (x)^{\mathrm{T}} \in \mathbb{R}^{2}$, $w \in \mathbb{R}^{m}$, $A \in \mathbb{R}^{2 \times 2}$ and $c \in \mathbb{R}^2$.

\textbf{Gradient:}

\begin{equation*}
    \nabla_s f_{\theta}(s) = \nabla_s a^{L}(s) \, w + A s + c
\end{equation*}

We set

\begin{equation*}
    D^{l} = \mathrm{diag} \left( \frac{\mathrm{d}}{\mathrm{d}z^{l}} \sigma_{l} (z^{l}) \right) \in \mathbb{R}^{m \times m}, \quad D_{i, i}^{l} = {\sigma_{l}}^{\prime} (z_{i}^{l})
\end{equation*}

\begin{align*}
    \nabla_s a^{L}(s) \, w & = \delta^{[1]}  \\
    \delta^{[1]} & = {W^{[1]}}^{\mathrm{T}} D^{[1]} \, \delta^{[2]} \\
    \delta^{[2]} & = \delta^{[3]} + h \, {W^{[2]}}^{\mathrm{T}} D^{[2]} \, \delta^{[3]} \\
    &\vdots\\
    \delta^{l} & = \delta^{[l+1]} + h \, {W^{l}}^{\mathrm{T}} D^{l} \, \delta^{[l+1]} \\
    &\vdots\\
    \delta^{L-1} & = \delta^{L} + h \, {W^{L-1}}^{\mathrm{T}} D^{L-1} \, \delta^{L} \\
    \delta^{L} & = w + h \, {W^{L}}^{\mathrm{T}} D^{L} \, w
\end{align*}


\textbf{Hessian:}

First my derivation:

Ruthotto computes the Laplacian of the potential model with respect to the spatial variable $x$. For this he uses the trace of the Hessian matrix of our model $f_{\theta}(s)$, i.e. 

\begin{align*}
    \Delta_x f_{\theta}(s) & = \mathrm{tr}(E^{\mathrm{T}} \, \nabla^{2}_s f_{\theta}(s) \, E) = \\
    & = \mathrm{tr}(E^{\mathrm{T}} \, (\nabla^{2}_s a^{L}(s) \, w + A) \, E) = \\
    & = \mathrm{tr}(E^{\mathrm{T}} \, \nabla^{2}_s a^{L}(s) \, w \, E) + \mathrm{tr}(E^{\mathrm{T}} \,  A \, E) = \\
    & = \Delta_x (a^{L}(s) \, w) + \mathrm{tr}(E^{\mathrm{T}} \,  A \, E)
\end{align*}

where the columns of $E \in \mathbb{R}^{(d+1) \times d}$ are given by the first $d$ standard basis vectors in $R^{d+1}$ so that only the second derivative of the spatial coordinates $x \in \mathbb{R}^d$ is computed. In our case we use $E = (1, 0)^{\mathrm{T}} \in \mathbb{R}^{2}$. \\
Ruthotto sets

\begin{align*}
    \Delta_x (a^{L}(s) \, w) & = t^{[0]} + h \, \sum^{L}_{l=1} t^{l} = \mathrm{tr}(\eta^{[0]}) + h \, \sum^{L}_{l=1} \mathrm{tr}(\eta^{l}) = \\
    & = \mathrm{tr} \left( \eta^{[0]} + h \, \sum^{L}_{l=1} \eta^{l} \right) = \mathrm{tr} \left( \nabla^{2}_x a^{L}(s) \, w \right)
\end{align*}

\begin{equation*}
    \partial_{xx} a^{L}(s) w = \nabla^{2}_x a^{L}(s) w = E^{\mathrm{T}} \, \nabla^{2}_s a^{L}(s) \, w \, E = \eta^{[0]} + h \, \sum^{L}_{l=1} \eta^{l}
\end{equation*}

where 

\begin{equation*}
    \eta^{[0]} = E^T {W^{[1]}}^{\mathrm{T}} \mathrm{diag}({\sigma_{[1]}}^{\prime \prime}(W^{[1]} s + b^{[1]}) \odot \delta^{[2]}) W^{[1]} E
\end{equation*}

\begin{equation*}
    \eta^{l} = {J^{l-1}}^{\mathrm{T}} {W^{l}}^{\mathrm{T}} \mathrm{diag}({\sigma_{l}}^{\prime \prime}(W^{l} a^{l-1}(s) + b^{l}) \odot \delta^{[l+1]}) W^{l} J^{l-1}
\end{equation*}

where 

\begin{equation*}
    J^{l-1} = J_x(a^{l-1}(s)) = \left( \frac{\mathrm{d}}{\mathrm{d} x} a^{l-1}_1(s), \ldots, \frac{\mathrm{d}}{\mathrm{d} x} a^{l-1}_m(s) \right)^{\mathrm{T}} \in \mathbb{R}^{m}
\end{equation*}


Now to the Hessian as it has been implemented: 

\begin{align*}
    \eta^{[1]} & = h \, {W^{[1]}}^{\mathrm{T}} \mathrm{diag} \left( H^{[1]} \delta^{[2]} \right) W^{[1]} + \\
    & \quad + \left( I + h \, {W^{[1]}}^{\mathrm{T}} D^{[1]} \right) \, \left( \left( I + h \, {W^{[1]}}^{\mathrm{T}} D^{[1]} \right) \, \eta^{[2]} \right)^{\mathrm{T}} \\ 
    &\vdots\\
    \eta^{l} & = h \, {W^{l}}^{\mathrm{T}} \mathrm{diag} \left( H^{l} \delta^{[l+1]} \right) W^{l} + \\
    & \quad + \left( I + h \, {W^{l}}^{\mathrm{T}} D^{l} \right) \, \left( \left( I + h \, {W^{l}}^{\mathrm{T}} D^{l} \right) \, \eta^{[l+1]} \right)^{\mathrm{T}} \\ 
    \eta^{l} & = h \, {W^{l}}^{\mathrm{T}} \mathrm{diag} \left( H^{l} \delta^{[l+1]} \right) W^{l} + \\
    & \quad + \left( I + h \, {W^{l}}^{\mathrm{T}} D^{l} \right) \, \eta^{[l+1]}  \left( I + h \,  D^{l} {W^{l}} \right) \\ 
    &\vdots\\
    \eta^{L-1} & = h \, {W^{L-1}}^{\mathrm{T}} \mathrm{diag} \left( H^{L-1} \delta^{L} \right) W^{L-1} + \\
    & \quad + \left( I + h \, {W^{L-1}}^{\mathrm{T}} D^{L-1} \right) \, \left( \left( I + h \, {W^{L-1}}^{\mathrm{T}} D^{L-1} \right) \, \eta^{L} \right)^{\mathrm{T}} \\
    \eta^{L} &  = h \, {W^{L}}^{\mathrm{T}} \mathrm{diag}(H^{L} w) W^{L} 
\end{align*}

