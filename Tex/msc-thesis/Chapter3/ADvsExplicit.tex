\section{Automatic differentiation vs explicit derivatives}

\chapter{Backpropagation}

In this chapter we want to specify an iterative scheme that implements the explicit partial derivatives of the loss function.

\begin{itemize}
    \item residual loss term for each edge $e \in \mathcal{E}$ \begin{equation*} \phi_{e,r} (X_e) \coloneqq \frac{1}{n_e} \sum_{i=1}^{n_e} (r_{\theta_e} (X_e^i))^2 \end{equation*} where \begin{equation*} r_{\theta_e} (X_e^i) = r_{\theta_e} (x_e^i, t_e^i) = \partial_t \rho_{\theta_{e}} (x_{e}^i , t_{e}^i) - \varepsilon \partial_{xx} \rho_{\theta_e}(x_e^i, t_e^i) - \partial_x [f(\rho_{\theta_e}(x_e^i, t_e^i)) \partial_x V_e(x_e^i, t_e^i)] \end{equation*} for $X_e = \{X_e^i\}_{i=1}^{n_e} = \{(t_e^i, x_e^i)\}_{i=1}^{n_e} \subset [0,\ell_e] \times (0, T)$.
    \item a misfit term enforcing the initial conditions for each edge $e \in \mathcal{E}$ \begin{equation*} \phi_{e,0} (X_{e,0}) \coloneqq \frac{1}{n_0} \sum_{i=1}^{n_0} (\rho_{\theta_e} (0,x_{e,0}^i) - \rho_{e,0}(x_{e,0}^i))^2, \end{equation*} where $X_{e,0} = \{x_{e,0}^i\}_{i=1}^{n_0} \subset [0, \ell_e]$ is a set of boundary points along $t=0$ and $\rho_{e,0}$ \ldots
    \item a misfit term enforcing the Kirchhoff-Neumann conditions for each vertex $v \in \mathcal{V}_\mathcal{K}$ \begin{equation*} \phi_{v,K} (X_{v,b}) \coloneqq \frac{1}{n_b} \sum_{i=1}^{n_b} (\sum_{e \in \mathcal{E}_v} (- \varepsilon \partial_x \rho_{\theta_e} (t_{v,b}^i, v) + f(\rho_{\theta_e} (t_{v,b}^i, v)) \partial_x V_e(t_{v,b}^i, v)) \, n_e (v))^2, \end{equation*} where $X_{v,b} = \{t_{v,b}^i\}_{i=1}^{n_b} \subset (0,T)$ is a set of time snapshots where the Kirchhoff-Neumann conditions are enforced, $\mathcal{E}_v$ is the set of edges incident to node $v$ and the derivative is taken into the outgoing direction,
    \item a misfit term enforcing the flux boundary conditions for each vertex $v \in \mathcal{V}_\mathcal{D}$ \begin{align*} \phi_{v,D} (X_{v,b}) & \coloneqq \frac{1}{n_b} \sum_{i=1}^{n_b} ( \sum_{e\in \mathcal{E}_v} (-\varepsilon \partial_x \rho_{\theta_e} (t_{v,b}^i, v) + f(\rho_{\theta_e} (t_{v,b}^i, v)) \partial_x V_e(t_{v,b}^i, v)) \, n_e (v) + \\ & \quad + \alpha_v(t_{v,b}^i) (1-\rho_v^i) - \beta_v(t_{v,b}^i) \rho_v^i)^2 \end{align*}
    \item a misfit term enforcing the continuity in the vertices $v \in \mathcal{V}_\mathcal{K}$ \begin{equation*} \phi_{v,c} (X_{v,b}) \coloneqq \frac{1}{\abs{\mathcal{E}_v}} \sum_{e \in \mathcal{E}_v} \frac{1}{n_b} \sum_{i=1}^{n_b} ( \rho_{\theta_e} (t_{v,b}^i, v) - \rho_{v}^i)^2, \end{equation*} with $X_{v,b}$ as introduced before. Here, we introduce for each $v \in \mathcal{V}_\mathcal{K}$ some additional variables $\{\rho_{v}^i\}_{i=1}^{n_b}$, that are appended to our parameter set $\theta$ and determined through the optimization process in our model.
\end{itemize}

\begin{align*} \label{eq:loss}
    \Phi(X_{data}) & =  \frac{1}{\abs{\mathcal{V}_\mathcal{D}}} \sum_{v \in \mathcal{V}_\mathcal{D}} \phi_{v,D}(X_{v,b}) + \frac{1}{\abs{\mathcal{V}_\mathcal{K}}} \sum_{v \in \mathcal{V}_\mathcal{K}} \big( \phi_{v,K} (X_{v,b}) + \phi_{v,c}(X_{v,b}) \big) + \\
    & \quad + \frac{1}{\abs{\mathcal{E}}} \sum_{e \in \mathcal{E}} \big( \phi_{e,r} (X_{e,r}) + \phi_{e,0} (X_{e,0}) \big). 
\end{align*}

\subsection{Backpropagation in FNN}

\begin{equation*}
    a^{[0]}(x,t) = \begin{pmatrix} x \\ t \end{pmatrix} \in \mathbb{R}^2
\end{equation*}

\begin{equation*}
    a^{[l]}(x,t) = \sigma^{[l]} (z^{[l]}(x,t)) = \sigma^{[l]} (W^{[l]} a^{[l-1]}(x,t) + b^{[l]}) \in \mathbb{R}^{n_l}, 
\end{equation*}

where $W^{[l]} \in \mathbb{R}^{n_l \times n_{l-1}}, \; b^{[l]} \in \mathbb{R}^{n_l}$ and $z^{[l]}(x,t) = W^{[l]} a^{[l-1]}(x,t) + b^{[l]} \in \mathbb{R}^{n_l}$ for $l = 1, \ldots, L$.

\begin{equation*}
    \rho_{\theta_e}(x, t) = a^{[L]}(x,t) = \sigma^{[L]} (W^{[L]} a^{[L-1]}(x,t) + b^{[L]}) \in \mathbb{R}^{1} 
\end{equation*}

\begin{align*}
    \partial_x \rho_{\theta_e}(x, t) & =  \partial_x a^{[L]}(x,t) = \partial_x \sigma^{[L]} (W^{[L]} a^{[L-1]}(x,t) + b^{[L]}) = \\
    & =  \frac{\mathrm{d}}{\mathrm{d} z^{[L]}} \sigma^{[L]} (z^{[L]}(x,t)) \, \cdot \, \frac{\mathrm{d}}{\mathrm{d} a^{[L]}} z^{[L]}(x,t) = \\ 
    & =  \frac{\mathrm{d}}{\mathrm{d} z^{[L]}} \sigma^{[L]} (z^{[L]}(x,t)) \, \cdot \, W^{[L]} \, \cdot \, \frac{\mathrm{d}}{\mathrm{d} x} a^{[L-1]}(x,t) = \\
    & =  \ldots = \\
    & =  \frac{\mathrm{d}}{\mathrm{d} z^{[L]}} \sigma^{[L]} (z^{[L]}(x,t)) \, \cdot \, W^{[L]} \, \cdot \, \frac{\mathrm{d}}{\mathrm{d} z^{[L-1]}} \sigma^{[L-1]} (z^{[L-1]}(x,t)) \, \cdot \, W^{[L-1]} \, \cdot \, \ldots \, \cdot \\
    & \quad \cdot \, \frac{\mathrm{d}}{\mathrm{d} z^{[1]}} \sigma^{[1]} (z^{[1]}(x,t)) \, \cdot \, W^{[1]} \, \cdot \,\frac{\mathrm{d}}{\mathrm{d} x} a^{[0]}(x,t) 
\end{align*}
    
\begin{equation*}
    \frac{\mathrm{d}}{\mathrm{d} x} a^{[0]}(x,t) = \frac{\mathrm{d}}{\mathrm{d} x} \begin{pmatrix} x \\ t \end{pmatrix} = \begin{pmatrix} 1 \\ 0 \end{pmatrix}
\end{equation*}

\begin{equation*}
    \frac{\mathrm{d}}{\mathrm{d} z^{[l]}} \sigma^{[l]} (z^{[l]}(x,t)) = D^{[l]} \in \mathbb{R}^{n_l \times n_l}, \quad D_{i, i}^{[l]} = {\sigma^{[l]}}^{\prime} (z_{i}^{[l]})
\end{equation*}

(diagonal matrix, since we apply the activation function component-wise)

\begin{equation*}
    \partial_x \rho_{\theta_e}(x, t) = D^{[L]} \, \cdot \, W^{[L]} \, \cdot \, D^{[L-1]} \, \cdot \, W^{[L-1]} \, \cdot \, \ldots \, \cdot \, D^{[1]} \, \cdot \, W^{[1]} \, \cdot \, \begin{pmatrix} 1 \\ 0 \end{pmatrix} \in \mathbb{R}
\end{equation*}

\begin{equation*}
    \partial_t \rho_{\theta_e}(x, t) = D^{[L]} \, \cdot \, W^{[L]} \, \cdot \, D^{[L-1]} \, \cdot \, W^{[L-1]} \, \cdot \, \ldots \, \cdot \, D^{[1]} \, \cdot \, W^{[1]} \, \cdot \, \begin{pmatrix} 0 \\ 1 \end{pmatrix} 
\end{equation*}

If we set $s = (x, t)^{\mathrm{T}}$ we can write the gradient of $\rho_{\theta_e}(s)$ as

\begin{equation*}
    \nabla_s \rho_{\theta_e}(s) = {W^{[1]}}^{\mathrm{T}} \, \cdot \, D^{[1]} \, \cdot \, {W^{[2]}}^{\mathrm{T}} \, \cdot \, D^{[2]} \, \cdot \, \ldots \, \cdot \, {W^{[L]}}^{\mathrm{T}} \, \cdot \, D^{[L]},
\end{equation*}

which we compute in the Backpropagation algorithm and get

\begin{equation*}
    {\nabla_s \rho_{\theta_e}(s)}^{\mathrm{T}} \cdot \begin{pmatrix} 1 \\ 0 \end{pmatrix} = \partial_x \rho_{\theta_e}(x, t) \quad \text{and} \quad {\nabla_s \rho_{\theta_e}(s)}^{\mathrm{T}} \cdot \begin{pmatrix} 0 \\ 1 \end{pmatrix} = \partial_t \rho_{\theta_e}(x, t)
\end{equation*}


\begin{equation*}
    \partial_x [f(\rho_{\theta_e}(x, t)) \partial_x V_e(x, t)] = \frac{\mathrm{d}}{\mathrm{d} \rho_{\theta_e}} f(\rho_{\theta_e}(x, t)) \partial_x \rho_{\theta_e}(x, t) \partial_x V_e(x, t) + f(\rho_{\theta_e}(x, t)) \partial_{xx} V_e(x, t)
\end{equation*}

I suppose we use $f(\rho_e) = \rho_e(1-\rho_e) \Rightarrow f^{\prime}(\rho_e) = 1 - 2 \rho_e$. Of course, $\partial_x V_e(x, t)$ and $\partial_{xx} V_e(x, t)$ still remain to be computed. 



\begin{equation*}
    \partial_{x x} \rho_{\theta_e}(x, t) = ???
\end{equation*}

Attempt 1: Suppose $\partial_x \rho_{\theta_e}(x, t)$ is computed correctly and the activation function on each layer is $L-1$ times continuously differentiable

\begin{align*}
    \partial_{x x} \rho_{\theta_e}(x, t) & = \partial_{x} \partial_{x} \rho_{\theta_e}(x, t) = \\
    & = \partial_{x} [D^{[L]} \, \cdot \, W^{[L]} \, \cdot \, D^{[L-1]} \, \cdot \, W^{[L-1]} \, \cdot \, \ldots \, \cdot \, D^{[1]} \, \cdot \, W^{[1]} \, \cdot \, (1, 0)^{\mathrm{T}}]
\end{align*}

Only $D^{[1]}, \ldots, D^{[L]}$ depend on $x$. We first take the derivative $D^{[1]}$ with respect to $x$. Since all $D^{[l]}$ are diagonal matrices, we only consider the diagonal elements $D_{i, i}^{[l]} = {\sigma^{[l]}}^{\prime} (z_{i}^{[l]}) \in \mathbb{R}$.  

\begin{align*}
    \frac{\mathrm{d}}{\mathrm{d} x} {\sigma^{[1]}}^{\prime} (z_{i}^{[1]}) & = \frac{\mathrm{d}}{\mathrm{d} z_{i}^{[1]}} {\sigma^{[1]}}^{\prime} (z_{i}^{[1]}) \frac{\mathrm{d}}{\mathrm{d} x} z_{i}^{[1]} = {\sigma^{[1]}}^{\prime \prime} (z_{i}^{[1]}) \frac{\mathrm{d}}{\mathrm{d} x} (W^{[1]} a^{[0]}(x,t) + b^{[1]})_i = \\
    & = {\sigma^{[1]}}^{\prime \prime} (z_{i}^{[1]}) \frac{\mathrm{d}}{\mathrm{d} x} \left[ \sum^{n_0}_{j=1} W^{[1]}_{i,j} a^{[0]}_{j}(x,t) + b^{[1]}_i \right] = \\
    & = {\sigma^{[1]}}^{\prime \prime} (z_{i}^{[1]}) \sum^{n_0}_{j=1} W^{[1]}_{i,j} \frac{\mathrm{d}}{\mathrm{d} x} a^{[0]}_{j}(x,t) = \\
    & = {\sigma^{[1]}}^{\prime \prime} (z_{i}^{[1]}) \left\langle W^{[1]}_{i, \colon}, \frac{\mathrm{d}}{\mathrm{d} x} a^{[0]}(x,t)  \right\rangle = \\
    & = {\sigma^{[1]}}^{\prime \prime} (W^{[1]}_{i, 1} + b^{[1]}_i) W^{[1]}_{i, 1}
\end{align*}

Okay, that wasn't bad. You can certainly continue it iteratively. 

But what about this long matrix product? Does the product rule apply here? If so, then we have to apply it $L-1$ times? And then we have to differentiate a selected matrix (probably $D^{[L]}$) $L-1$ times. Great.

$\Rightarrow$ If we consider the matrix product component-wise, then the product rule must actually be applied $L-1 \times n_{L-1} \times n_{L-2} \times \ldots \times n_{2}$ times for $\frac{\mathrm{d}}{\mathrm{d} z^{[l]}_i} \sigma^{[l]}(z^{[l]}_i)$ in the second derivative.  

\textbf{Ruthotto:}

For $n_L = 1$ we have

\begin{align*}
    \delta^{[L]} & = {W^{[L]}}^{\mathrm{T}} \mathrm{diag}({\sigma^{[L]}}^{\prime}(W^{[L]} a^{[L-1]}(s) + b^{[L]})) = {W^{[L]}}^{\mathrm{T}} D^{[L]} \in \mathbb{R}^{n_{L-1}} \\
    \delta^{[l]} & = {W^{[l]}}^{\mathrm{T}} D^{[l]} \cdot \, \delta^{[l+1]} \, \cdot \, \ldots \,  \cdot \, \delta^{[L]} \in \mathbb{R}^{n_{l-1}} \quad \text{for} \quad l = 2, \ldots, L-1
\end{align*}

We note that for $A \in \mathbb{R}^{m \times n}$, $b \in \mathbb{R}^{m}$, $c, d \in \mathbb{R}^{n}$

\begin{align*}
    d \, \odot \, c &= c \, \odot \, d \in \mathbb{R}^{n} \\
    \mathrm{diag}(b) \, \cdot \, A &= b \, \odot \, A \in \mathbb{R}^{m \times n} \\
    A \, \cdot \, \mathrm{diag}(c) &= A \, \odot \, c^{\mathrm{T}} \in \mathbb{R}^{m \times n} \\
    \mathrm{diag}(b) \, \cdot \, A \, \cdot \, \mathrm{diag}(c) &= b \, \odot \, A \, \odot \, c^{\mathrm{T}} \in \mathbb{R}^{m \times n} \\
    \mathrm{diag}(c) \, \cdot \, d &= c \, \odot \, d \in \mathbb{R}^{n} \\
    c^{\mathrm{T}} \, \cdot \, \mathrm{diag}(d) &= c^{\mathrm{T}} \, \odot \, d^{\mathrm{T}} \in \mathbb{R}^{1 \times n} \\
\end{align*}

We set 

\begin{equation*}
    H^{[l]} = \mathrm{diag}({\sigma^{[l]}}^{\prime \prime}(W^{[l]} a^{[l-1]}(s) + b^{[l]})) \in \mathbb{R}^{n_l \times n_l}, \quad (H^{[l]})_{i, i} = {\sigma^{[l]}}^{\prime \prime} (z_{i}^{[l]})
\end{equation*}

\begin{align*}
    \nabla^{2} \rho_{\theta_e}(s) & = \vartheta^{[1]} \in \mathbb{R}^{2 \times 2} \\
    \vartheta^{[1]} & = {W^{[1]}}^{\mathrm{T}} \left( \mathrm{diag}(H^{[1]} \delta^{[2]}) + D^{[1]} \vartheta^{[2]} D^{[1]} \right) W^{[1]} \\
    & \vdots \\
    \vartheta^{[l]} & = {W^{[l]}}^{\mathrm{T}} \left( \mathrm{diag}(H^{[l]} \delta^{[l+1]}) + D^{[l]} \vartheta^{[l+1]} D^{[l]} \right) W^{[l]} \in \mathbb{R}^{n_{l-1} \times n_{l-1}} \\
    & \vdots \\
    \vartheta^{[L]} & = {W^{[L]}}^{\mathrm{T}} H^{[L]} W^{[L]} \in \mathbb{R}^{n_{L-1} \times n_{L-1}}
\end{align*}



\textbf{Derivative of the loss function $\Phi$ with respect to model parameters $\theta_e$}

Since $\Phi$ is a sum of various other loss functions, we consider the derivatives of the individual loss functions $\phi_{e,r}, \phi_{e,0}, \phi_{v,K}, \phi_{v,D}, \phi_{v,c}$. 

\begin{equation*}
    \frac{\mathrm{d}}{\mathrm{d} \theta_e} \phi_{e,r} = \frac{2}{n_e} \sum_{i=1}^{n_e} r_{\theta_e} (X_e^i) \frac{\mathrm{d}}{\mathrm{d} \theta_e} r_{\theta_e} (X_e^i) 
\end{equation*}

\begin{equation*}
    \frac{\mathrm{d}}{\mathrm{d} \theta_e} r_{\theta_e} (X_e^i) =  \frac{\mathrm{d}}{\mathrm{d} \theta_e} \partial_t \rho_{\theta_{e}} (x_{e}^i , t_{e}^i) - \varepsilon \frac{\mathrm{d}}{\mathrm{d} \theta_e} \partial_{xx} \rho_{\theta_e}(x_e^i, t_e^i) - \frac{\mathrm{d}}{\mathrm{d} \theta_e} \partial_x [f(\rho_{\theta_e}(x_e^i, t_e^i)) \partial_x V_e(x_e^i, t_e^i)] 
\end{equation*}

\begin{equation*}
    \frac{\mathrm{d}}{\mathrm{d} \theta_e} \phi_{e,0} (X_{e,0}) = \frac{2}{n_0} \sum_{i=1}^{n_0} (\rho_{\theta_e} (0,x_{e,0}^i) - \rho_{e,0}(x_{e,0}^i)) \frac{\mathrm{d}}{\mathrm{d} \theta_e} \rho_{\theta_e} (0,x_{e,0}^i)
\end{equation*}

\begin{align*}
    \frac{\mathrm{d}}{\mathrm{d} \theta_e} \phi_{v,K} (X_{v,b}) & = \frac{2}{n_b} \sum_{i=1}^{n_b} (\sum_{e \in \mathcal{E}_v} (- \varepsilon \partial_x \rho_{\theta_e} (t_{v,b}^i, v) + f(\rho_{\theta_e} (t_{v,b}^i, v)) \partial_x V_e(t_{v,b}^i, v)) \, n_e (v)) \cdot \\ 
    & \quad \cdot ((- \varepsilon \frac{\mathrm{d}}{\mathrm{d} \theta_e} \partial_x \rho_{\theta_e} (t_{v,b}^i, v) + \frac{\mathrm{d}}{\mathrm{d} \theta_e} f(\rho_{\theta_e} (t_{v,b}^i, v)) \partial_x V_e(t_{v,b}^i, v)) \, n_e (v))
\end{align*}

\begin{align*}
    \frac{\mathrm{d}}{\mathrm{d} \theta_e} \phi_{v,D} (X_{v,b}) & = \frac{2}{n_b} \sum_{i=1}^{n_b} ( \sum_{e\in \mathcal{E}_v} (-\varepsilon \partial_x \rho_{\theta_e} (t_{v,b}^i, v) + f(\rho_{\theta_e} (t_{v,b}^i, v)) \partial_x V_e(t_{v,b}^i, v)) \, n_e (v) + \\ 
    & \quad + \alpha_v(t_{v,b}^i) (1-\rho_v^i) - \beta_v(t_{v,b}^i) \rho_v^i )  \cdot \\
    & \quad \cdot ((-\varepsilon \frac{\mathrm{d}}{\mathrm{d} \theta_e} \partial_x \rho_{\theta_e} (t_{v,b}^i, v) + \frac{\mathrm{d}}{\mathrm{d} \theta_e} f(\rho_{\theta_e} (t_{v,b}^i, v)) \partial_x V_e(t_{v,b}^i, v)) \, n_e (v))
\end{align*}

\begin{equation*}
    \frac{\mathrm{d}}{\mathrm{d} \theta_e} \phi_{v,c} (X_{v,b}) \frac{2}{\abs{\mathcal{E}_v}} \sum_{e \in \mathcal{E}_v} \frac{1}{n_b} \sum_{i=1}^{n_b} ( \rho_{\theta_e} (t_{v,b}^i, v) - \rho_{v}^i) \frac{\mathrm{d}}{\mathrm{d} \theta_e} \rho_{\theta_e} (t_{v,b}^i, v)
\end{equation*}

Derivatives that I still have to calculate:

\begin{equation*}
    \frac{\mathrm{d}}{\mathrm{d} \theta_e} \rho_{\theta_e}(x, t) \Rightarrow \textbf{AD}
\end{equation*}

\begin{equation*}
    \frac{\mathrm{d}}{\mathrm{d} \theta_e} \partial_x \rho_{\theta_e}(x, t) \Rightarrow \textbf{AD}
\end{equation*}

\begin{equation*}
    \frac{\mathrm{d}}{\mathrm{d} \theta_e} \partial_t \rho_{\theta_e}(x, t) \Rightarrow \textbf{AD}
\end{equation*}

\begin{equation*}
    \frac{\mathrm{d}}{\mathrm{d} \theta_e} \partial_{xx} \rho_{\theta_e}(x, t) \Rightarrow \textbf{AD}
\end{equation*}

\begin{equation*}
    \frac{\mathrm{d}}{\mathrm{d} \theta_e} f(\rho_{\theta_e}(x, t)) = \frac{\mathrm{d}}{\mathrm{d} \theta_e} \rho_{\theta_e}(x, t) - 2 \frac{\mathrm{d}}{\mathrm{d} \theta_e} \rho_{\theta_e}(x, t)
\end{equation*}

\begin{align*}
    \frac{\mathrm{d}}{\mathrm{d} \theta_e} & [\frac{\mathrm{d}}{\mathrm{d} \rho_{\theta_e}} f(\rho_{\theta_e}(x, t)) \partial_x \rho_{\theta_e}(x, t) \partial_x V_e(x, t)] = \\
    \\ \quad & = \frac{\mathrm{d}}{\mathrm{d} {\rho_{\theta_e}}^2} f(\rho_{\theta_e}(x, t)) \frac{\mathrm{d}}{\mathrm{d} \theta_e} \rho_{\theta_e}(x, t) \partial_x \rho_{\theta_e}(x, t) \partial_x V_e(x, t) + \\
    \\ \quad & + \frac{\mathrm{d}}{\mathrm{d} \rho_{\theta_e}} f(\rho_{\theta_e}(x, t)) \frac{\mathrm{d}}{\mathrm{d} \theta_e} \partial_x \rho_{\theta_e}(x, t) \partial_x V_e(x, t)
\end{align*}



\subsection{Backpropagation in ResNet}

\begin{equation*}
    a^{[1]}(x,t) = \sigma^{[1]} (W^{[1]} a^{[0]}(x,t) + b^{[1]}) \in \mathbb{R}^{m}, 
\end{equation*}

where $W^{[1]} \in \mathbb{R}^{m \times 2}$ and $b^{[1]} \in \mathbb{R}^{m}$.

\begin{equation*}
    a^{[l]}(x,t) = a^{[l-1]}(x,t) + h \, \sigma^{[l]} (z^{[l]}(x,t)) = a^{[l-1]}(x,t) + h \, \sigma^{[l]} (W^{[l]} a^{[l-1]}(x,t) + b^{[l]}) \in \mathbb{R}^{m}, 
\end{equation*}

for $l = 2, \ldots, L$, where $W^{[2]}, \ldots, W^{[L]} \in \mathbb{R}^{m \times m}$ and $b^{[2]}, \ldots, b^{[L]} \in \mathbb{R}^{m}$.

\begin{equation*}
    \rho_{\theta_e}(x, t) = \rho_{\theta_e}(s) = w^{\mathrm{T}} a^{[L]}(s) + \frac{1}{2} s^{\mathrm{T}} A s + c^{\mathrm{T}} s 
\end{equation*}

where $s = (x, t)^{\mathrm{T}} \in \mathbb{R}^{2}$, $w \in \mathbb{R}^{m}$, $A \in \mathbb{R}^{2 \times 2}$ and $c \in \mathbb{R}^2$.

\textbf{Gradient:}

\begin{equation*}
    \nabla_s \rho_{\theta_e}(s) = \nabla_s a^{[L]}(s) \, w + A s + c
\end{equation*}

We set

\begin{equation*}
    D^{[l]} = \mathrm{diag} \left( \frac{\mathrm{d}}{\mathrm{d}z^{[l]}} \sigma^{[l]} (z^{[l]}) \right) \in \mathbb{R}^{m \times m}, \quad D_{i, i}^{[l]} = {\sigma^{[l]}}^{\prime} (z_{i}^{[l]})
\end{equation*}

\begin{align*}
    \nabla_s a^{[L]}(s) \, w & = \delta^{[1]}  \\
    \delta^{[1]} & = {W^{[1]}}^{\mathrm{T}} D^{[1]} \, \delta^{[2]} \\
    \delta^{[2]} & = \delta^{[3]} + h \, {W^{[2]}}^{\mathrm{T}} D^{[2]} \, \delta^{[3]} \\
    &\vdots\\
    \delta^{[l]} & = \delta^{[l+1]} + h \, {W^{[l]}}^{\mathrm{T}} D^{[l]} \, \delta^{[l+1]} \\
    &\vdots\\
    \delta^{[L-1]} & = \delta^{[L]} + h \, {W^{[L-1]}}^{\mathrm{T}} D^{[L-1]} \, \delta^{[L]} \\
    \delta^{[L]} & = w + h \, {W^{[L]}}^{\mathrm{T}} D^{[L]} \, w
\end{align*}


\textbf{Hessian:}

First my derivation:

Ruthotto computes the Laplacian of the potential model with respect to the spatial variable $x$. For this he uses the trace of the Hessian matrix of our model $\rho_{\theta_e}(s)$, i.e. 

\begin{align*}
    \Delta_x \rho_{\theta_e}(s) & = \mathrm{tr}(E^{\mathrm{T}} \, \nabla^{2}_s \rho_{\theta_e}(s) \, E) = \\
    & = \mathrm{tr}(E^{\mathrm{T}} \, (\nabla^{2}_s a^{[L]}(s) \, w + A) \, E) = \\
    & = \mathrm{tr}(E^{\mathrm{T}} \, \nabla^{2}_s a^{[L]}(s) \, w \, E) + \mathrm{tr}(E^{\mathrm{T}} \,  A \, E) = \\
    & = \Delta_x (a^{[L]}(s) \, w) + \mathrm{tr}(E^{\mathrm{T}} \,  A \, E)
\end{align*}

where the columns of $E \in \mathbb{R}^{(d+1) \times d}$ are given by the first $d$ standard basis vectors in $R^{d+1}$ so that only the second derivative of the spatial coordinates $x \in \mathbb{R}^d$ is computed. In our case we use $E = (1, 0)^{\mathrm{T}} \in \mathbb{R}^{2}$. \\
Ruthotto sets

\begin{align*}
    \Delta_x (a^{[L]}(s) \, w) & = t^{[0]} + h \, \sum^{L}_{l=1} t^{[l]} = \mathrm{tr}(\vartheta^{[0]}) + h \, \sum^{L}_{l=1} \mathrm{tr}(\vartheta^{[l]}) = \\
    & = \mathrm{tr} \left( \vartheta^{[0]} + h \, \sum^{L}_{l=1} \vartheta^{[l]} \right) = \mathrm{tr} \left( \nabla^{2}_x a^{[L]}(s) \, w \right)
\end{align*}

\begin{equation*}
    \partial_{xx} a^{[L]}(s) w = \nabla^{2}_x a^{[L]}(s) w = E^{\mathrm{T}} \, \nabla^{2}_s a^{[L]}(s) \, w \, E = \vartheta^{[0]} + h \, \sum^{L}_{l=1} \vartheta^{[l]}
\end{equation*}

where 

\begin{equation*}
    \vartheta^{[0]} = E^T {W^{[1]}}^{\mathrm{T}} \mathrm{diag}({\sigma^{[1]}}^{\prime \prime}(W^{[1]} s + b^{[1]}) \odot \delta^{[2]}) W^{[1]} E
\end{equation*}

\begin{equation*}
    \vartheta^{[l]} = {J^{[l-1]}}^{\mathrm{T}} {W^{[l]}}^{\mathrm{T}} \mathrm{diag}({\sigma^{[l]}}^{\prime \prime}(W^{[l]} a^{[l-1]}(s) + b^{[l]}) \odot \delta^{[l+1]}) W^{[l]} J^{[l-1]}
\end{equation*}

where 

\begin{equation*}
    J^{[l-1]} = J_x(a^{[l-1]}(s)) = \left( \partial_x a^{[l-1]}_1(s), \ldots, \partial_x a^{[l-1]}_m(s) \right)^{\mathrm{T}} \in \mathbb{R}^{m}
\end{equation*}


Now to the Hessian as it has been implemented: 

\begin{align*}
    \vartheta^{[1]} & = h \, {W^{[1]}}^{\mathrm{T}} \mathrm{diag} \left( H^{[1]} \delta^{[2]} \right) W^{[1]} + \\
    & \quad + \left( I + h \, {W^{[1]}}^{\mathrm{T}} D^{[1]} \right) \, \left( \left( I + h \, {W^{[1]}}^{\mathrm{T}} D^{[1]} \right) \, \vartheta^{[2]} \right)^{\mathrm{T}} \\ 
    &\vdots\\
    \vartheta^{[l]} & = h \, {W^{[l]}}^{\mathrm{T}} \mathrm{diag} \left( H^{[l]} \delta^{[l+1]} \right) W^{[l]} + \\
    & \quad + \left( I + h \, {W^{[l]}}^{\mathrm{T}} D^{[l]} \right) \, \left( \left( I + h \, {W^{[l]}}^{\mathrm{T}} D^{[l]} \right) \, \vartheta^{[l+1]} \right)^{\mathrm{T}} \\ 
    \vartheta^{[l]} & = h \, {W^{[l]}}^{\mathrm{T}} \mathrm{diag} \left( H^{[l]} \delta^{[l+1]} \right) W^{[l]} + \\
    & \quad + \left( I + h \, {W^{[l]}}^{\mathrm{T}} D^{[l]} \right) \, \vartheta^{[l+1]}  \left( I + h \,  D^{[l]} {W^{[l]}} \right) \\ 
    &\vdots\\
    \vartheta^{[L-1]} & = h \, {W^{[L-1]}}^{\mathrm{T}} \mathrm{diag} \left( H^{[L-1]} \delta^{[L]} \right) W^{[L-1]} + \\
    & \quad + \left( I + h \, {W^{[L-1]}}^{\mathrm{T}} D^{[L-1]} \right) \, \left( \left( I + h \, {W^{[L-1]}}^{\mathrm{T}} D^{[L-1]} \right) \, \vartheta^{[L]} \right)^{\mathrm{T}} \\
    \vartheta^{[L]} &  = h \, {W^{[L]}}^{\mathrm{T}} \mathrm{diag}(H^{[L]} w) W^{[L]} 
\end{align*}

